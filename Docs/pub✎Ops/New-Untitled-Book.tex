% Created 2020-10-15 Thu 15:32
% Intended LaTeX compiler: pdflatex
\documentclass[11pt]{article}
\usepackage[utf8]{inputenc}
\usepackage[T1]{fontenc}
\usepackage{graphicx}
\usepackage{grffile}
\usepackage{longtable}
\usepackage{wrapfig}
\usepackage{rotating}
\usepackage[normalem]{ulem}
\usepackage{amsmath}
\usepackage{textcomp}
\usepackage{amssymb}
\usepackage{capt-of}
\usepackage{hyperref}
\author{Alisha Awen - Harmonic Alchemy Productions}
\date{\textit{<2020-07-30 Thu>}}
\title{HAP - Book Project Skeleton}
\hypersetup{
 pdfauthor={Alisha Awen - Harmonic Alchemy Productions},
 pdftitle={HAP - Book Project Skeleton},
 pdfkeywords={},
 pdfsubject={HAP - Bare Bones Skeleton with generic basics already done for.  Great for Authors to use when starting a new book project from scratch\ldots{}},
 pdfcreator={Emacs 28.0.50 (Org mode 9.3)}, 
 pdflang={English}}
\begin{document}

\maketitle
\tableofcontents

To use this Book Project Skeleton (template) for your own projects, perform the following three steps:

\begin{enumerate}
\item \textbf{Clone a Copy of this File:} Put it where you normally organize your writing projects.  Place it within its own folder named after your intended project title\ldots{}

\item \textbf{Copy/Clone ./media \uline{(and all its contents)}:} To your new project's folder \uline{(where you copied this file)}\ldots{}  Any external documents or images referenced by your book project should be placed within this ./media/ directory\ldots{} All of the external links, images, within the boilerplate examples may be used as examples for getting your own images, and external attachments linked properly\ldots{}  This single ./media directory serves media and external attachments for all of the .org files contained within this .emacs.d/Docs/pub✎Ops directory\ldots{} Cloning any of the other .org files will also require copy/cloning the ./media directory relative to them\ldots{}

\item \textbf{Start Writing Your Book:}  Change headings and structure to match your books outline Sections/Chapters/Scenes/ Plots, Characters, etc. Start doing the Tasks within the Tasks section\ldots{} Have fun doing this\ldots{}
\end{enumerate}

The In-Buffer Settings and KEYWORDS specified above are configured to get this file looking good when exported as a Standard PDF file:  To make a PDF out of this file issue the following Emacs command:

M-x org-latex-export-to-pdf

After AucTeX is done parsing/compiling etc., a new PDF by the name of New-Untitled-Book.pdf -and- New-Untitled-Book.tex will appear within the same directory as this file\ldots{}  

Subsequent repeated calls to the above command will overwrite your existing New-Untitled-Book.pdf -and- New-Untitled-Book.tex files! 

Change the \#+EXPORT\textsubscript{FILE}\textsubscript{NAME}: directive at the top of this file to "Your-Book's-Title.pdf" That will be the file name produced thence forward\ldots{}

\textit{<2020-07-30 Thu> } So far things are pretty basic PDF\ldots{} We will get fancy with other formats later\ldots{} I promice\ldots{} I have to do all this documentation first!

Enjoy! - Alisha Awen

\section{💡 Scratchpad Refile:}
\label{sec:org62d4ac2}
\subsection{☞ Start Here!}
\label{sec:org775416c}

This Section is mainly to accommodate "seat-of-pants" writers (of which I have been accused of but I also crave the planning stuff).  I often come here for my \href{https://projects.csail.mit.edu/gsb/old-archive/gsb-archive/gsb2000-02-11.html}{"yak shaving"} adventures\ldots{}

Use this: "💡 Scratchpad Refile ☞ Start Here!" section as a quick place to document ideas and inspirations as they come alone.  You can refile them later\ldots{} (even if it is intended for elsewhere) Parallel universe? Sure\ldots{} You got that covered. \%\^{})

\begin{itemize}
\item Use this section to quickly capture inspired at-the-moment ideas While they are still fresh in mind.

\item Come directly here:

\begin{itemize}
\item Advance your cursor a few newlines below this README drawer on a fresh clean line with some extra space below it as well\ldots{} (above any previously time stamped entries you may already have placed)\ldots{} If this is the first time, get rid of the "Visisonar - Object" example entry below\ldots{} That is only for the template to illustrate these instructions by example.

\item You are now at the top of the list, (a few lines below this README drawer). The time is the present. -and- You are all fired up with an idea to write about!

\item Enter a new timestamp with "C-c .", (that's: Press Ctrl-c, let go, and then type a Period .)  A date chooser panel will pop up\ldots{}

\item Press ENTER to auto accept today's date and time.  A new timestamped line will appear with your cursor blinking on the right of it\ldots{} There won't be any future entries above you\ldots{} (maybe next decade? time travel?)

\item Advance your cursor a few more lines down to some fresh empty space -and-

\item Start Typing Your Idea Like Mad!

\item Don't stop and don't worry about formatting or organizing it until later, after all your inspiration has passed and you are ready to organize\ldots{}
\end{itemize}

\item After You are Done Writing:

\begin{itemize}
\item Come back to the timestamp line where you first started\ldots{} make sure the timestamp has at least one empty line above and below it.

\item Give the timestamp line a title to the right of the timestamp.

\item Make this timestamped line an outline heading with: "C-c ENTER" (that's Press Ctrl-c, let go and then press ENTER)
(make sure this new heading has empty lines above and below it)

\item Adjust this new outline heading level with: M-▷ (this will make it a child of ☞ Start Here! heading above, rather than a sibling).

\item Adjust the space between your heading and your fantastic idea below it to one single blank line.

\item You can organize everything else later\ldots{}  Don't scare away the muses! Get your story down now!

\item Then you may go out to play elsewhere little grasshopper\ldots{} But come back later and organize things\ldots{} OK? Don't create a swamp in here! And don't forget to take your bath tonight\ldots{} ;-)

\item Having done the simple steps above any time you get inspired, you will be able to tell your grand children.. "I got my first spark of an idea for that book back in Year, Month, Day, time, second" (50 years later!) Now that is being organized! LOL
\end{itemize}
\end{itemize}

\subsubsection{\textit{<2020-06-07 Sun> } Visisonar - Object from Asimov's Foundation Series}
\label{sec:org6948d4c}

This is a dummy example of some crazy idea that you may have gotten at the spur of the moment, while you were doing something elsewhere within this project\ldots{}  After you are done getting as much of the idea as you initally can then you can simply go back to what you were doing before the inspiration segue tore you away\ldots{} you might want to schedule this as a todo item before leaving though\ldots{}

Later when you are free you can come back to file this or place it where it belongs.  In the case of the note below it would probably go in a research folder of a Sci-Fi related project\ldots{}

From Isaac Asimov's science fiction series 'Foundation', specifically the section of 'Foundation and Empire' entitled 'The Mule', the visisonar is considered a musical instrument.

Unlike most instruments, this one does not achieve its effect by using sonic effects but, instead, directly stimulates the listeners' mental well-being. The number of people and level of effect depends on the skill level of the player. As this was a device from the First Empire, there were not that many people around with any skill at all. However, the Mule's clown, Magnifico, had some skill with the instrument and was asked to play for select parties on the Foundation and for large groups of workers on Haven. This may not have been a good idea\ldots{}
\section{📖 Book:}
\label{sec:orgc1cd57c}
\subsection{🔖 Chapter 1}
\label{sec:orgf4a9b31}

\subsubsection{📄 Episode - <replace w/ chapter name>}
\label{sec:orgf7a33fb}
Episodes or Scenes are the DNA that make up the chapters in a novel\ldots{}  

Chapters can contain several scenes or episodes\ldots{} There can be many twisty paths all alike (or all different)\ldots{} \%\textasciitilde{})

<insert chapter / episode text here>

Lorem Ipsum Lorem Ipsum dolor sit amet, consectetuer adipiscingelit. Duis tellus. Donec ante dolor, iaculis nec, gravidaac, cursus in, eros. Mauris vestibulum, felis et egestasullamcorper, purus nibh vehicula sem, eu egestas antenisl non justo. Fusce tincidunt, lorem nev dapibusconsectetuer, leo orci mollis ipsum, eget suscipit erospurus in ante. 

At ipsum vitae est lacinia tincidunt. Maecenas elit orci,gravida ut, molestie non, venenatis vel, lorem. Sedlacinia. Suspendisse potenti. Sed ultricies cursuslectus. In id magna sit amet nibh suspicit euismod.Integer enim. Donec sapien ante, accumsan ut,sodales commodo, auctor quis, lacus. Maecenas a elitlacinia urna posuere sodales. Curabitur pede pede,molestie id, blandit vitae, varius ac, purus. Mauris atipsum vitae est lacinia tincidunt. Maecenas elit orci, gravida ut, molestie non, venenatis vel,lorem. Sed lacinia. Suspendisse potenti. Sed ultrucies cursus lectus. In id magna sit amet nibhsuspicit euismod. Integer enim. Donec sapien ante, accumsan ut, sodales commodo, auctorquis, lacus. Maecenas a elit lacinia urna posuere sodales. Curabitur pede pede, molestie id,blandit vitae, varius ac, purus.

\subsubsection{🗒 Background Notes:}
\label{sec:org4a42c11}

\begin{enumerate}
\item 🗒 Note 1: \textit{<2020-05-10 Sun>}
\label{sec:orgc87d1b0}

\begin{itemize}
\item Mark each note's timestamp to its initial creation time. This will enable emerging notes for chapters to be searched chronologically, and to give you an idea of how things evolve over time\ldots{}

\item Add anything you need here under this heading formatted any way needed to present the data\ldots{} Sub levels are fine, tables are fine\ldots{} etc.  This is your chapter's Notebook, Scrapbook, whatever any extra info/metadata you feel is important to record\ldots{}

\begin{itemize}
\item Use bulleted lists/sublist etc. if needed\ldots{}
\end{itemize}

\item Or Numbered Lists\ldots{}
\end{itemize}

\begin{center}
\begin{tabular}{rr}
\hline
Or & Tables\\
\hline
1 & 2.3\\
\hline
\end{tabular}
\end{center}

Do any or all the above to help get the concepts and images of your book clear in mind\ldots{}

\item 🗒 Note 2: \textit{<2020-05-10 Sun>}
\label{sec:org83cea48}

Each new note needs to get its own heading and initial timestamp\ldots{}

\begin{enumerate}
\item Quid Novi?
\label{sec:orgabbc26f}

Quid Novi? Lorem Ipsum dolor sit amet,consectetur adipisicing elit, sed doeiusmod tempor incididunt ut laboreet dolore magna aliqua. Ut enim adminim veniam, quis nostrudexercitation ullamco laboris nisi utaliquip ex ea commodo consequat.Duis aute irure dolor inreprehenderit in coluptate velit essecillum dolore eu fugiat nulla pariatur.Excepteur sint occaecat cupidatatnon proident, sunt in culpa quiofficia deserunt mollit anim id estlaborum.
\end{enumerate}

\item 🗒 Phasellus orci: \textit{<2020-05-10 Sun>}
\label{sec:org793ef59}

Etiam tempor elit auctor magna. Nullam nibh velit, vestibulum ut, eleifend non, pulvinar eget, enim. Classaptent taciti sociosqu ad litora torquent per conubia nostra, per inceptos hymenaeos. Integer velit mauris, convallis acongue sed, placerat id, odio. Etiam venenatis tortor sed lectus. Nulla non orci. In egestas porttitor quam. Duis nec diameget nibh mattis tempus. Curabitus accumsan pede id odio. Nunc vitae libero. Aenean condimentum diam et turpis.Vestibulum non risus. Ut consectetuer gravida elit. Aenean est nunc, varius sed, alquam eu, feugiat sit amet, metus. Sedvenenatis odio id eros.

\begin{center}
\begin{tabular}{lllll}
\hline
Inceptos & Venenatis & Convallis & Curabitus & Nunc vitae libero\\
\hline
test 1 & test 2 & test 3 & test 4 & test 5\\
\hline
\end{tabular}
\end{center}

Yadda, yadda, yadda. etc\ldots{}
\end{enumerate}



\subsection{🔖 Chapter 2}
\label{sec:orgd3319af}

\subsubsection{📄 Episode - <replace w/ chapter name>}
\label{sec:org9af18c4}
Episodes or Scenes are the DNA that make up the chapters in a novel\ldots{}  

Chapters can contain several scenes or episodes\ldots{} There can be many twisty paths all alike (or all different)\ldots{} \%\textasciitilde{})

<insert chapter / episode text here>

Lorem Ipsum Lorem Ipsum dolor sit amet, consectetuer adipiscingelit. Duis tellus. Donec ante dolor, iaculis nec, gravidaac, cursus in, eros. Mauris vestibulum, felis et egestasullamcorper, purus nibh vehicula sem, eu egestas antenisl non justo. Fusce tincidunt, lorem nev dapibusconsectetuer, leo orci mollis ipsum, eget suscipit erospurus in ante. 

At ipsum vitae est lacinia tincidunt. Maecenas elit orci,gravida ut, molestie non, venenatis vel, lorem. Sedlacinia. Suspendisse potenti. Sed ultricies cursuslectus. In id magna sit amet nibh suspicit euismod.Integer enim. Donec sapien ante, accumsan ut,sodales commodo, auctor quis, lacus. Maecenas a elitlacinia urna posuere sodales. Curabitur pede pede,molestie id, blandit vitae, varius ac, purus. Mauris atipsum vitae est lacinia tincidunt. Maecenas elit orci, gravida ut, molestie non, venenatis vel,lorem. Sed lacinia. Suspendisse potenti. Sed ultrucies cursus lectus. In id magna sit amet nibhsuspicit euismod. Integer enim. Donec sapien ante, accumsan ut, sodales commodo, auctorquis, lacus. Maecenas a elit lacinia urna posuere sodales. Curabitur pede pede, molestie id,blandit vitae, varius ac, purus.

\subsubsection{🗒 Background Notes:}
\label{sec:orge71537c}

\begin{enumerate}
\item 🗒 Note 1: \textit{<2020-05-10 Sun>}
\label{sec:org1b228ff}

\begin{itemize}
\item Mark each note's timestamp to its initial creation time. This will enable emerging notes for chapters to be searched chronologically, and to give you an idea of how things evolve over time\ldots{}

\item Add anything you need here under this heading formatted any way needed to present the data\ldots{} Sub levels are fine, tables are fine\ldots{} etc.  This is your chapter's Notebook, Scrapbook, whatever any extra info/metadata you feel is important to record\ldots{}

\begin{itemize}
\item Use bulleted lists/sublist etc. if needed\ldots{}
\end{itemize}

\item Or Numbered Lists\ldots{}
\end{itemize}

\begin{center}
\begin{tabular}{rr}
\hline
Or & Tables\\
\hline
1 & 2.3\\
\hline
\end{tabular}
\end{center}

Do any or all the above to help get the concepts and images of your book clear in mind\ldots{}

\item 🗒 Note 2: \textit{<2020-05-10 Sun>}
\label{sec:orgba91a20}

Each new note needs to get its own heading and initial timestamp\ldots{}

\begin{enumerate}
\item Quid Novi?
\label{sec:orge9a91e2}

Quid Novi? Lorem Ipsum dolor sit amet,consectetur adipisicing elit, sed doeiusmod tempor incididunt ut laboreet dolore magna aliqua. Ut enim adminim veniam, quis nostrudexercitation ullamco laboris nisi utaliquip ex ea commodo consequat.Duis aute irure dolor inreprehenderit in coluptate velit essecillum dolore eu fugiat nulla pariatur.Excepteur sint occaecat cupidatatnon proident, sunt in culpa quiofficia deserunt mollit anim id estlaborum.
\end{enumerate}

\item 🗒 Phasellus orci: \textit{<2020-05-10 Sun>}
\label{sec:orga6139ce}

Etiam tempor elit auctor magna. Nullam nibh velit, vestibulum ut, eleifend non, pulvinar eget, enim. Classaptent taciti sociosqu ad litora torquent per conubia nostra, per inceptos hymenaeos. Integer velit mauris, convallis acongue sed, placerat id, odio. Etiam venenatis tortor sed lectus. Nulla non orci. In egestas porttitor quam. Duis nec diameget nibh mattis tempus. Curabitus accumsan pede id odio. Nunc vitae libero. Aenean condimentum diam et turpis.Vestibulum non risus. Ut consectetuer gravida elit. Aenean est nunc, varius sed, alquam eu, feugiat sit amet, metus. Sedvenenatis odio id eros.

\begin{center}
\begin{tabular}{lllll}
\hline
Inceptos & Venenatis & Convallis & Curabitus & Nunc vitae libero\\
\hline
test 1 & test 2 & test 3 & test 4 & test 5\\
\hline
\end{tabular}
\end{center}

Yadda, yadda, yadda. etc\ldots{}
\end{enumerate}



\subsection{🔖 Chapter 3}
\label{sec:org5f6d88e}

\subsubsection{📄 Episode - <replace w/ chapter name>}
\label{sec:orgf482ffb}
Episodes or Scenes are the DNA that make up the chapters in a novel\ldots{}  

Chapters can contain several scenes or episodes\ldots{} There can be many twisty paths all alike (or all different)\ldots{} \%\textasciitilde{})

<insert chapter / episode text here>

Lorem Ipsum Lorem Ipsum dolor sit amet, consectetuer adipiscingelit. Duis tellus. Donec ante dolor, iaculis nec, gravidaac, cursus in, eros. Mauris vestibulum, felis et egestasullamcorper, purus nibh vehicula sem, eu egestas antenisl non justo. Fusce tincidunt, lorem nev dapibusconsectetuer, leo orci mollis ipsum, eget suscipit erospurus in ante. 

At ipsum vitae est lacinia tincidunt. Maecenas elit orci,gravida ut, molestie non, venenatis vel, lorem. Sedlacinia. Suspendisse potenti. Sed ultricies cursuslectus. In id magna sit amet nibh suspicit euismod.Integer enim. Donec sapien ante, accumsan ut,sodales commodo, auctor quis, lacus. Maecenas a elitlacinia urna posuere sodales. Curabitur pede pede,molestie id, blandit vitae, varius ac, purus. Mauris atipsum vitae est lacinia tincidunt. Maecenas elit orci, gravida ut, molestie non, venenatis vel,lorem. Sed lacinia. Suspendisse potenti. Sed ultrucies cursus lectus. In id magna sit amet nibhsuspicit euismod. Integer enim. Donec sapien ante, accumsan ut, sodales commodo, auctorquis, lacus. Maecenas a elit lacinia urna posuere sodales. Curabitur pede pede, molestie id,blandit vitae, varius ac, purus.

\subsubsection{🗒 Background Notes:}
\label{sec:org08e4187}

\begin{enumerate}
\item 🗒 Note 1: \textit{<2020-05-10 Sun>}
\label{sec:orgab5e6e1}

\begin{itemize}
\item Mark each note's timestamp to its initial creation time. This will enable emerging notes for chapters to be searched chronologically, and to give you an idea of how things evolve over time\ldots{}

\item Add anything you need here under this heading formatted any way needed to present the data\ldots{} Sub levels are fine, tables are fine\ldots{} etc.  This is your chapter's Notebook, Scrapbook, whatever any extra info/metadata you feel is important to record\ldots{}

\begin{itemize}
\item Use bulleted lists/sublist etc. if needed\ldots{}
\end{itemize}

\item Or Numbered Lists\ldots{}
\end{itemize}

\begin{center}
\begin{tabular}{rr}
\hline
Or & Tables\\
\hline
1 & 2.3\\
\hline
\end{tabular}
\end{center}

Do any or all the above to help get the concepts and images of your book clear in mind\ldots{}

\item 🗒 Note 2: \textit{<2020-05-10 Sun>}
\label{sec:org84e0ced}

Each new note needs to get its own heading and initial timestamp\ldots{}

\begin{enumerate}
\item Quid Novi?
\label{sec:org00c10a2}

Quid Novi? Lorem Ipsum dolor sit amet,consectetur adipisicing elit, sed doeiusmod tempor incididunt ut laboreet dolore magna aliqua. Ut enim adminim veniam, quis nostrudexercitation ullamco laboris nisi utaliquip ex ea commodo consequat.Duis aute irure dolor inreprehenderit in coluptate velit essecillum dolore eu fugiat nulla pariatur.Excepteur sint occaecat cupidatatnon proident, sunt in culpa quiofficia deserunt mollit anim id estlaborum.
\end{enumerate}

\item 🗒 Phasellus orci: \textit{<2020-05-10 Sun>}
\label{sec:orgeb11b43}

Etiam tempor elit auctor magna. Nullam nibh velit, vestibulum ut, eleifend non, pulvinar eget, enim. Classaptent taciti sociosqu ad litora torquent per conubia nostra, per inceptos hymenaeos. Integer velit mauris, convallis acongue sed, placerat id, odio. Etiam venenatis tortor sed lectus. Nulla non orci. In egestas porttitor quam. Duis nec diameget nibh mattis tempus. Curabitus accumsan pede id odio. Nunc vitae libero. Aenean condimentum diam et turpis.Vestibulum non risus. Ut consectetuer gravida elit. Aenean est nunc, varius sed, alquam eu, feugiat sit amet, metus. Sedvenenatis odio id eros.

\begin{center}
\begin{tabular}{lllll}
\hline
Inceptos & Venenatis & Convallis & Curabitus & Nunc vitae libero\\
\hline
test 1 & test 2 & test 3 & test 4 & test 5\\
\hline
\end{tabular}
\end{center}

Yadda, yadda, yadda. etc\ldots{}
\end{enumerate}



\subsection{📒 Back Story Plots:}
\label{sec:org50afa94}

\subsubsection{Plot Example One:}
\label{sec:orge5526e2}

Note: This is an example of a back story plot related to your book\ldots{} Write it down in the rough here and then later you can develop it further if needed.

\subsubsection{Plot Example Two:}
\label{sec:orgf88871b}

Note: This is an example of a back story plot related to your book\ldots{} Write it down in the rough here and then later you can develop it further if needed.

\subsubsection{Add more Plots like this:}
\label{sec:orgde83826}

Note: This is an example of a back story plot related to your book\ldots{} Write it down in the rough here and then later you can develop it further if needed.

\subsection{🗡 Chapter Plot Summaries:}
\label{sec:orgd996eb9}

Gather all the plot ideas that have been building over the years and put them in here\ldots{} then you can decide how and when to present them within the unfolding story.

\subsubsection{Chapter \# Plots}
\label{sec:org3ca7808}

In this chapter we are dealing with  yadda yadda yadda which must be resolved with yadda yadda yadda.

\subsubsection{Chapter \# Plots}
\label{sec:orgb78f7e9}

In this chapter we are dealing with  yadda yadda yadda which must be resolved with yadda yadda yadda.

\subsubsection{Chapter \# Plots}
\label{sec:org320dff3}

In this chapter we are dealing with  yadda yadda yadda which must be resolved with yadda yadda yadda.
\section{📒 Research:}
\label{sec:orgd41fc0e}
\subsection{💡 New Episode Ideas:}
\label{sec:org91ae945}
\subsection{👤 Characters:}
\label{sec:org0c194c3}
\subsection{👥 Groups - Factions:}
\label{sec:orgd937162}
\subsection{🌐 Locations - Worlds:}
\label{sec:org3bc19a5}
\subsection{🎪 Major Events:}
\label{sec:org81a50c3}
\subsection{🏞 Scenes:}
\label{sec:org845490e}
\subsection{❝ Quotations:}
\label{sec:org924fe8e}
\subsection{🖍 Styles:}
\label{sec:org381762a}
\end{document}
