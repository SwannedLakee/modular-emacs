%%%% ~~~~~~~~~~~~~~~~~~~~~~~~~~~~~~~~~~~~~~~~~~~~~~~~~~~~~~~~~~~~~~~~~~~~~~~~~~~~
%%   File:        ~/.emacs.d/Docs/TeX/tufte-handout-setup.tex
%%   Ref:         <https://github.com/harmonicalchemy/modular-emacs>
%%   Purpose:     LaTeX Setup Include File for Tufte-book class classy books
%%   Author:      Alisha Awen
%%   Maintainer:  Alisha Awen
%%   Created:     2025-006-22
%%   Updated:     2025-006-22
%%
%%   INFO: Edward Tufte is known for graphical excellence in his famous books.
%%   Some enthusiasts combined his design principles into LaTeX and you have the
%%   tufte-book and tufte-handout classes for excellence in typesetting...
%%   THIS File Typesets to: tuftebook Classy Book Style...

%%   This program is free software: you can redistribute it and/or modify
%%   it under the terms of the GNU General Public License as published by
%%   the Free Software Foundation, either version 3 of the License, or
%%   (at your option) any later version.

%%   This program is distributed in the hope that it will be useful,
%%   but WITHOUT ANY WARRANTY; without even the implied warranty of
%%   MERCHANTABILITY or FITNESS FOR A PARTICULAR PURPOSE.  See the
%%   GNU General Public License for more details.

%%   You should have received a copy of the GNU General Public License
%%   along with this program.  If not, see <https://www.gnu.org/licenses/>.

%%   To Customize: Copy/Clone ~/.emacs.d/Docs/PubOps/org-templates (which contains
%%   this file) to your .org doc's MASTER FOLDER. Then, modify your cloned copy of
%%   this file...

%%   ORG-MODE [DEFAULT-PACKAGES]: Do NOT \usepackage ANY of the Packages
%%      Listed HERE:  (org-mode already loaded these by default GLOBALLY)
%%      See: 09-4-org-export-conf.el
%%      DEFAULT PACKAGES: 
%%
%%         inputenc, fontenc, graphicx, longtable, wrapfig
%%         rotating, ulem, amsmath, amssymb, capt-of, hyperref
%%
%%   DEBUGGING TIP:
%%      USE unix command line tool: "kpsewhich" to see if a package is installed
%%      (similar to the unix command: which)
%%      Example:
%%      $ kpsewhich article.cls
%%        /opt/local/share/texmf-texlive/tex/latex/base/article.cls
%%      $ kpsewhich notes.cls
%%        (no output)
%%      Results from Above:
%%        The file article.cls was installed by macports texlive
%%        There is NO installed package called notes.cls 
%%%% ~~~~~~~~~~~~~~~~~~~~~~~~~~~~~~~~~~~~~~~~~~~~~~~~~~~~~~~~~~~~~~~~~~~~~~~~~~~~

%%% ~~~~~~~~~~~~~~~~~~~~~~~~~~~~~~~~~
%%  GENERAL TUFTE CLASS PACKAGES:
%%% ~~~~~~~~~~~~~~~~~~~~~~~~~~~~~~~~~

\usepackage{color}
%\usepackage{amssymb}  %% Already Provided by org-mode DEFAULT-PACKAGES
%\usepackage{amsmath} %% Already Provided by org-mode DEFAULT-PACKAGES
\usepackage{gensymb}
\usepackage{nicefrac}
\usepackage{units}"

%%% ~~~~~~~~~~~~~~~~
%%  BETTER SOURCE CODE LISTINGS:
%%  NOTE: the minted package is ALREADY LOADED by:
%%        09-4-org-export-conf.el Just USE it HERE...

%% Use Console Command:
%%    $ pygmentize -L styles
%% to see a list of all possible "minted" code listing styles.
%% Then change {style-name} below to try any of them out...
%%    tango   - Typical...
%%    perldoc - A Nice One...
%%    xcode   - Similar to perldoc (even nicer)

\setminted{style=xcode,frame=leftline}
\setminted[r]{linenos=true}

%%% ~~~~~~~~~~~~~~~~~~~~~~~~~~~~~~~~~
%%  LANDSCAPE & MARGINS:
%%% ~~~~~~~~~~~~~~~~~~~~~~~~~~~~~~~~~

%% This allows using \begin{landscape} & \end{landscape} 
%% Before and After WIDE Figures, Tables, images, etc...
\usepackage{pdflscape}

%%% ~~~~~~~~~~~~~~~~~~~~~~~~~~~~~~~~~
%% FONTS:
%% NOTE: [T1]{fontenc} is included and LOADED
%%       by org-mode [DEFAULT-PACKAGES]
%%       (org-latex-default-packages-alist)
%%% ~~~~~~~~~~~~~~~~~~~~~~~~~~~~~~~~~

%% Currently using the DEFAULT fonts memoir provides out of box...

%%% ~~~~~~~~~~~~~~~~~~~~~~~~~~~~~~~~~
%%. TABLES
%%% ~~~~~~~~~~~~~~~~~~~~~~~~~~~~~~~~~

%% Book Like Table Formatting
\usepackage{booktabs} 

%% Set table column padding to be proportional to text width
\setlength{\tabcolsep}{0.0075\textwidth}

%%% ~~~~~~~~~~~~~~~~~~~~~~~~~~~~~~~~~
%% DEFAULT PARAGRAPH SETTINGS %%
%% Yeah... I read the manual and their reasons for indenting paragraphs,
%% BUT this is a LOG/Journal... NOT a Fiction Novel...
%%% ~~~~~~~~~~~~~~~~~~~~~~~~~~~~~~~~~

%%% ~~~~~~~~~~~~~~~~
%% BLOCK PARAGRAPHS No Indent:

\setlength{\parindent}{0pt} % Paragraph indentation
\setlength{\parskip}{6pt} % Vertical space between paragraphs

%%% ~~~~~~~~~~~~~~~~~~~~~~~~~~~~~~~~~
%% SPECIAL PACKAGES:
%%% ~~~~~~~~~~~~~~~~~~~~~~~~~~~~~~~~~

%%% ~~~~~~~~~~~~~~~~
%%  GRAPHICS PACKAGES: 

\usepackage{svg}

%%% ~~~~~~~~~~~~~~~~
%%  MUSIC PACKAGES: 

\usepackage[minimal]{leadsheets}
\useleadsheetslibraries{musicsymbols}
\useleadsheetslibraries{chords}

%%% ~~~~~~~~~~~~~~~~~~~~~~~~~~~~~~~~~
%%  FINAL PREAMBLE DIRECTIVES:
%%% ~~~~~~~~~~~~~~~~~~~~~~~~~~~~~~~~~


%%%% END: ~/.emacs.d/Docs/TeX/tufte-handout-setup.tex
%%%% ~~~~~~~~~~~~~~~~~~~~~~~~~~~~~~~~~~~~~~~~~~~~~~~~~~~~~~~~~~~~~~~~~~~~~~~~~~~~
